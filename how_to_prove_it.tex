\documentclass{scrartcl}

% --- PREAMBLE ---
% The amsmath package is essential for professional math typesetting.
% It gives us the align* environment and the \text{} command.
\usepackage{amsmath} 
% Provides better symbols, like \coloneqq for :=
\usepackage{amssymb} 
% For better margins
\usepackage[margin=1in]{geometry}
\usepackage{enumitem}

\begin{document}
    % Use a standard section heading for the problems
    \section*{1.4 Problems}

    % Use the 'enumerate' environment for a numbered list of problems.
    % It handles all numbering and indentation automatically.
    \begin{enumerate}[start=5]
        \item Verify the following identities by writing out (using logical symbols) what it means for an object x to be an element of each set and then using logical equivalences.
            % Use a nested enumerate for sub-problems (a), (b), etc.
            \begin{enumerate}
                \item $A \setminus (A \cap B) = A \setminus B$
                \item $A \cup (B \cap C) = (A \cup B) \cap (A \cup C)$
            \end{enumerate}

        \item[7.] Verify the following identities by writing out (using logical symbols) what it means for an object x to be an element of each set and then using logical equivalences.
            \begin{enumerate}
                \item $(A \cup B) \setminus C = (A \setminus C) \cup (B \setminus C)$
                \item $A \cup (B \setminus C) = (A \cup B) \setminus (C \setminus A)$
            \end{enumerate}
    
        \item[8.] Use any method you wish to verify the following identities:
            \begin{enumerate}
                \item $(A \setminus B) \cap C = (A \cap C) \setminus B$
                \item $(A \cap B) \setminus B = \varnothing$
                \item $A \setminus (A \setminus B) = A \cap B$
            \end{enumerate}
        \item[9.] For each of the following sets, write out (using logical symbols) what it means for an object x to be an element of the set. Then determine which of these sets must be equal to each other by determining which statements are equivalent.
            \begin{enumerate}
                \item $(A \setminus B) \setminus C$
                \item $A \setminus (B \setminus C)$
                \item $(A \setminus B) \cup (A \cap C)$
                \item $(A \setminus B) \cap (A \setminus C)$
                \item $A \setminus (B \cup C)$
            \end{enumerate}
    \end{enumerate}

    \section*{1.4 Solutions}

    \subsection*{Solution to 5(a)}
    To show $A \setminus (A \cap B) = A \setminus B$, we show that for any object $x$, the statement $x \in A \setminus (A \cap B)$ is logically equivalent to $x \in A \setminus B$.

    % Use the 'align*' environment for a series of equations or equivalences. 
    % The '&' indicates where LaTeX aligns each line. 
    % The '\' starts a new line. 
    % The '*' prevents automatic line numbering numbering.
    \begin{align*}
        x \in A \setminus (A \cap B) 
        % The \Leftrightarrow symbol is often used for logical equivalence.
        &\Leftrightarrow x \in A \wedge \neg (x \in A \cap B) 
        && \text{(by definition of $\setminus$)} \\
        % The \text{} command is used for adding commentary within a math environment.
        &\Leftrightarrow x \in A \wedge \neg (x \in A \wedge x \in B)
        && \text{(by definition of $\cap$)} \\
        &\Leftrightarrow x \in A \wedge (x \notin A \vee x \notin B)
        && \text{(by De Morgan's Law)} \\
        &\Leftrightarrow (x \in A \wedge x \notin A) \vee (x \in A \wedge x \notin B)
        && \text{(by the Distributive Law)} \\
        &\Leftrightarrow \bot \vee (x \in A \wedge x \notin B)
        && \text{(by the Law of Contradiction)} \\
        &\Leftrightarrow x \in A \wedge x \notin B
        && \text{(by the Identity Law for $\vee$)} \\
        &\Leftrightarrow x \in A \setminus B
        && \text{(by definition of $\setminus$)}
    \end{align*}
    \hfill $\blacksquare$


    \subsection*{Solution to 5(b)}
    To show $A \cup (B \cap C) = (A \cup B) \cap (A \cup C)$, we show that for any object $x$, the statement $x \in A \cup (B \cap C)$ is equivalent to $x \in (A \cup B) \cap (A \cup C)$.

    \begin{align*}
        x \in A \cup (B \cap C)
        &\Leftrightarrow x \in A \vee x \in (B \cap C)
        && \text{(by definition of $\cup$)} \\
        &\Leftrightarrow x \in A \vee (x \in B \wedge x \in C)
        && \text{(by definition of $\cap$)} \\
        &\Leftrightarrow (x \in A \vee x \in B) \wedge (x \in A \vee x \in C)
        && \text{(by the Distributive Law)} \\
        &\Leftrightarrow x \in (A \cup B) \wedge x \in (A \cup C)
        && \text{(by definition of $\cup$, applied twice)} \\
        &\Leftrightarrow x \in (A \cup B) \cap (A \cup C)
        && \text{(by definition of $\cap$)}
    \end{align*}
    \hfill $\blacksquare$

    \subsection*{Solution to 6(a)}
    To show $(A \cup B) \setminus C = (A \setminus C) \cup (B \setminus C)$, we show that for any object $x$, \\
    the statement $x \in (A \cup B) \setminus C$ is equivalent to $x \in (A \setminus C) \cup (B \setminus C)$.

    \begin{align*}
        x \in (A \cup B) \setminus C
        &\Leftrightarrow x \in (A \cup B) \wedge x \notin C 
        && \text{(by definition of $\setminus$)} \\
        &\Leftrightarrow (x \in A \vee x \in B) \wedge x \notin C
        && \text{(by definition of $\cup$)} \\
        &\Leftrightarrow  (x \in A \wedge x \notin C) \vee (x \in B \wedge x \notin C)
        && \text{(by the Distributive Law)} \\
        &\Leftrightarrow  x \in (A \setminus C) \vee x \in (B \setminus C)
        && \text{(by definition of $\setminus$, applied twice)} \\
        &\Leftrightarrow  x \in (A \setminus C) \cup (B \setminus C)
        && \text{(by definition of $\cup$)} \\
    \end{align*}
    \hfill $\blacksquare$

    \subsection*{Solution to 6(b)}
        To show $A \cup (B \setminus C) = (A \cup B) \setminus (C \setminus A)$, we show that for any object $x$, \\
        the statement $x \in A \cup (B \setminus C)$ is equivalent to $x \in (A \cup B) \setminus (C \setminus A)$.

    \begin{align*}
        x \in A \cup (B \setminus C)
        &\Leftrightarrow x \in A \vee x \in (B \setminus C)
        && \text{(by definition of $\cup$)} \\
        &\Leftrightarrow x \in A \vee (x \in B \wedge x \notin C)
        && \text{(by definition of $\setminus$)} \\
        &\Leftrightarrow (x \in A \vee x \in B) \wedge (x \in A \vee x \notin C)
        && \text{(by the Distributive Law )} \\
        &\Leftrightarrow (x \in A \vee x \in B) \wedge \neg (x \notin A \wedge x \in C)
        && \text{(by the De Morgan Law )} \\
        &\Leftrightarrow (x \in A \vee x \in B) \wedge \neg (x \in C \wedge x \notin A)
        && \text{(by the Commutative Law )} \\
        &\Leftrightarrow x \in (A \cup B) \wedge \neg (x \in C \wedge x \notin A)
        && \text{(by the definition of $\cup$ )} \\
        &\Leftrightarrow x \in (A \cup B) \wedge x \notin (C \setminus A)
        && \text{(by the definition of $\setminus$ )} \\
        &\Leftrightarrow x \in (A \cup B) \setminus (C \setminus A)
        && \text{(by the definition of $\setminus$ )} \\
    \end{align*}
    \hfill $\blacksquare$

    \subsection*{Solution to 8(a)}
    To show $(A \setminus B) \cap C = (A \cap C) \setminus B$, we show that for any object $x$, \\
    the statement $x \in (A \setminus B) \cap C$ is equivalent to $x \in (A \cap C) \setminus B$.

    \begin{align*}
        x \in (A \setminus B) \cap C
        &\Leftrightarrow x \in (A \setminus B) \wedge x \in C
        && \text{(by definition of $\cap$)} \\
        &\Leftrightarrow (x \in A \wedge x \notin B) \wedge x \in C
        && \text{(by definition of $\setminus$)} \\
        &\Leftrightarrow x \in A \wedge (x \notin B \wedge x \in C)
        && \text{(by the Associative Law)} \\
        &\Leftrightarrow x \in A \wedge (x \in C \wedge x \notin B)
        && \text{(by the Commutative Law)} \\
        &\Leftrightarrow (x \in A \wedge x \in C) \wedge x \notin B
        && \text{(by the Associative Law)} \\
        &\Leftrightarrow x \in (A \cap C) \wedge x \notin B
        && \text{(by definition of $\cap$)} \\
        &\Leftrightarrow x \in (A \cap C) \setminus B
        && \text{(by definition of $\setminus$)} \\
    \end{align*}
    \hfill $\blacksquare$

    \subsection*{Solution to 8(b)}
    To show $(A \cap B) \setminus B = \varnothing$, we show that for any object $x$, the statement $x \in (A \cap B) \setminus B$ \\ 
    is equivalent to $x \in \varnothing$.

    \begin{align*}
        x \in (A \cap B) \setminus B
        &\Leftrightarrow x \in (A \cap B) \wedge x \notin B
        && \text{(by definition of $\setminus$)} \\
        &\Leftrightarrow (x \in A \wedge x \in B) \wedge x \notin B
        && \text{(by definition of $\cap$)} \\
        &\Leftrightarrow x \in A \wedge (x \in B \wedge x \notin B)
        && \text{(by the Associative Law)} \\
        &\Leftrightarrow x \in A \wedge \bot
        && \text{(by the Contradiction Law)} \\
        &\Leftrightarrow x \in \bot
        && \text{(by the Contradiction Law)} \\
        &\Leftrightarrow \bot
        && \text{(by the Domination Law for $\wedge$)} \\
        &\Leftrightarrow x \in \varnothing
        && \text{(by definition of $\varnothing$)}
    \end{align*}
    \hfill $\blacksquare$


    \subsection*{Solution to 8(c)}
        To show $A \setminus (A \setminus B) = A \cap B$, we show that for any object $x$, the statement $x \in A \setminus (A \setminus B)$ \\ 
        is equivalent to $x \in A \cap B$.

    \begin{align*}
        x \in A \setminus (A \setminus B) 
        &\Leftrightarrow x \in A \wedge x \notin (A \setminus B)
        && \text{(by definition of $\setminus$  )} \\
        &\Leftrightarrow x \in A \wedge \neg (x \in A \wedge x \notin B)
        && \text{(by definition of $\setminus$)} \\
        &\Leftrightarrow x \in A \wedge (x \notin A \vee x \in B)
        && \text{(by the De Morgan Law)} \\
        &\Leftrightarrow (x \in A \wedge x \notin A) \vee (x \in A \wedge x \in B)
        && \text{(by the Distributive Law)} \\
        &\Leftrightarrow \bot \vee (x \in A \wedge x \in B)
        && \text{(by the Contradiction Law)} \\
        &\Leftrightarrow x \in A \wedge x \in B
        && \text{(by the Identity Law)} \\
        &\Leftrightarrow x \in A \cap B
        && \text{(by the definition of $\cap$)} \\
    \end{align*}
    \hfill $\blacksquare$
    \newpage

    \subsection*{Solution to 9(a)}
        $(A \setminus B) \setminus C$

    \subsection*{Solution to 9(b)}
        $A \setminus (B \setminus C)$

    \subsection*{Solution to 9(c)}
        $(A \setminus B) \cup (A \cap C)$

    \subsection*{Solution to 9(d)}
        $(A \setminus B) \cap (A \setminus C)$

    \subsection*{Solution to 9(e)}
        $A \setminus (B \cup C)$

\end{document}
